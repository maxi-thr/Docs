% !TeX root = CV.tex
\documentclass[a4paper]{cv_class}

\setname{Maximilian}{Dietlmeier}
\setaddress{München/ Deutschland}
\setmobile{01578/ 6224854}
\setmail{maxi.dietlmeier@web.de}
\setposition{Wissenschaftlicher Mitarbeiter} %ignored for now
%\setlinkedinaccount{https://www.linkedin.com/in/maximilian-dietlmeier-537b08173/} %you can play with color of the template (red is also nice..)
%\setgithubaccount{https://github.com/maxi-thr} %you can play with color of the template (red is also nice..)
\definecolor{maxi_blue}{HTML}{219C90}
\setthemecolor{maxi_blue} %you can play with color of the template (red is also nice..)

\begin{document}
%Set variables
%You can add sections, texts, explanations just by copying the style below. Replace the dummy texts "\lipsum[1][x-x]\par" with actual texts.
%Create header
\headerview
\vspace{2ex}
%Sections
%
% Summary
\addblocktext{Überblick}{
    \small
    Ich bin ein engagierter und ehrgeiziger Ingenieur mit einer großen Leidenschaft für Forschung, Entwicklung und Optimierung.  Dank meiner schnellen Auffassungsgabe und meines analytischen Denkvermögens kann ich mich effektiv in neue Themen einarbeiten und auch komplexe Herausforderungen erfolgreich meistern. So wurde ich beispielsweise im Rahmen meiner Masterarbeit zum Thema "Automatisierte Klassifizierung von Altholz mittels Deep-Learning-Techniken" mit dem AALE Student Award 2023 für die beste Masterarbeit ausgezeichnet.

    Durch meine beruflichen Erfahrungen als wissenschaftlicher Mitarbeiter konnte ich meine Problemlösungsfähigkeiten und meine Kompetenzen im interdisziplinären Arbeiten weiterentwickeln. Ich verfüge über fundierte Kenntnisse in Programmiersprachen wie Python, Julia und Matlab und habe besondere Stärken in der Datenverarbeitung und Algorithmenentwicklung. Ich bin in der Lage, effektiv im Team zu arbeiten und komplexe Sachverhalte klar zu vermitteln, da ich über ausgezeichnete kommunikative Fähigkeiten verfüge.
    
    Mein Ehrgeiz und Durchhaltevermögen, geprägt durch langjährige sportliche Aktivitäten, spiegeln sich in meiner Arbeitsweise wider. Ich bringe Leidenschaft und Engagement in jedes Projekt ein und freue mich darauf, in einer neuen Position meine Fähigkeiten einzusetzen, um gemeinsam innovative Lösungen zu entwickeln.
}
%%%%%%%%%%Old Executive Summary for Promotion%%%%%%%%%%%%%%%

% Ich bin eine zielstrebige und forschungsbegeisterte Person mit einem klaren Fokus auf Entwicklung und Optimierung. Direkt nach dem Abitur habe ich zielstrebig mein Studium begonnen und den Bachelor und Master erfolgreich abgeschlossen. Mein aktuelles Ziel ist es, mein Wissen im Bereich Data Science durch eine Promotion weiter zu vertiefen und auszubauen.

% Durch meine langjährigen sportlichen Aktivitäten habe ich einen starken Ehrgeiz und Durchhaltevermögen entwickelt, Eigenschaften, die sich in meiner beruflichen Laufbahn als sehr wertvoll erwiesen haben. Es war schon immer mein Drang, mir neues Wissen anzueignen, insbesondere durch Lesen und intensive Auseinandersetzung mit neuen Themen.
    
% In meiner bisherigen Tätigkeit als wissenschaftlicher Mitarbeiter konnte ich meine analytischen Fähigkeiten vertiefen und komplexe Prozessabläufe verstehen und optimieren. Darüber hinaus habe ich meine Programmierkenntnisse ausgebaut und bin versiert in der Erstellung von Berichten und der Präsentation von Forschungsergebnissen. Besonders viel Freude bereitet mir das interdisziplinäre Arbeiten, das ich sowohl während meiner Masterarbeit als auch an der TH Rosenheim schätzen gelernt habe. Diese Erfahrungen haben mich in meinem Wunsch bestärkt, meine Fähigkeiten im Rahmen einer Promotion weiter zu spezialisieren und zur Lösung anspruchsvoller wissenschaftlicher Fragestellungen beizutragen.


%Education
\section{Ausbildung} 

\datedexperience{Oskar-Maria-Graf Gymnasium}{2009-2017} 
\explanation{Allgemeines Abitur} 
\datedexperience{TH Landshut}{2017-2021} 
\explanation{B.Eng in Int. Wirtschaftsingenieurwesen, Abschlussnote 2.1} 
\explanationdetail{\coloredbullet\ %
    Bachelor Arbeit über 'Artificial Intelligence on Microcontrollers', Note 1.0\par}
\datedexperience{TH Rosenheim}{2021-2022} 
\explanation{M.Eng in Elektro- Informationstechnik, Abschlussnote 1.5}
\explanationdetail{\coloredbullet\ %
    Master Arbeit über 'Automated Classification of Postconsumer Wood with Rapid FLIM utilizing Deep Learning Techniques', Note 1.0\par}
   
%
% Experience
\section{Arbeitserfahrungen} 
    %
    \datedexperience{MSD International GmbH }{07-08/2017} 
    \explanation{Zeitarbeitskraft} 
    %
    \datedexperience{KiBATEC}{08-09/2017} 
    \explanation{Praktikum} 
    %
    \datedexperience{MSD International GmbH}{01/2018-07/2019} 
    \explanation{Werkstudent} 
    %
    \datedexperience{BMW Group, Landshut}{02/2020-08/2020} 
    \explanation{Praktikum} 
    %
    \datedexperience{TH Rosenheim}{10/2021-08/2022} 
    \explanation{Projekt Mitarbeiter für das Master Projekt und die Masterarbeit im FrIDAH Projekt}
    %
    \datedexperience{TH Rosenheim}{08/2022-04/2023}
    \explanation{Wissenschaftlicher Mitarbeiter im FrIDAH Projekt}
    %
    \datedexperience{TH Rosenheim}{04/2023-jetzt}
    \explanation{Wissenschaftlicher Mitarbeiter im Projekt Hochdynamische Antriebstechnik III}
    %\explanationdetail{\coloredbullet\ %
    % \lipsum[1][3-4]\par %replace this part with actual text
    % }
%
% Skills
\section{Kenntnisse}
    %
    \newcommand{\skillone}{\createskill{Programmiersprachen}{\textbf{\emph{Erfahren:}} \ \  Python \cpshalf Julia \cpshalf Matlab  \ \ \textbf{\emph{Vertraut:}} \ \  Bash \cpshalf C++ \cpshalf R}}
    %
    \newcommand{\skilltwo}{\createskill{Software}{Linux \cpshalf Microsoft Office \cpshalf GIT \cpshalf LaTeX}}
    %
    \newcommand{\skillthree}{\createskill{Interessensschwerpunkte}{Maschinelles Lernen \cpshalf Datenverarbeitung \cpshalf Algorithmenentwicklung}}
    %
    \newcommand{\skillfour}{\createskill{Sprachen}{\textbf{\emph{Muttersprache:}} \ \  Deutsch \ \ \textbf{\emph{Fließend:}} \ \ Englisch \ \ \textbf{\emph{Schulkenntnisse:}} \ \  Französisch \cpshalf Spanisch }}
    %
    \createskills{\skillone, \skilltwo, \skillthree, \skillfour}
%
% Experience
% \section{Extra}
%     \newcommand{\extraone}{%
%     \lipsum[1][7-8]\par %replace this part with actual text
%     }
%     %
%     \newcommand{\extratwo}{%
%     \lipsum[1][9-10]\par %replace this part with actual text
%     }
%     %
%     \newcommand{\extrathree}{%
%     \lipsum[1][11-12]%replace this part with actual text
%     }
%     %
%     \newcommand{\listofextras}{\extraone, \extratwo, \extrathree}
%     %
%     \createbullets{\listofextras}
%
%Footnote
%\createfootnote
\end{document}
